\documentclass{article}
\usepackage{amsmath}
\usepackage{amsthm}
\usepackage{forest}
\usepackage[makeroom]{cancel}
\begin{document}
	\section{Concepts}
	\subsection{Good proofs}
	Characteristics of good proofs are:
	\begin{itemize}
		\item 1) Correct
		\item 2) Complete
		\item 3) Clear
		\item 4) Brief
		\item 5) "Elegant" (clever, crisp)
		\item 6) Well organized (using lemmas as if you are using subroutines in code)
		\item 7) In order (from the most basic to the most complex, do it bottom-up)
	\end{itemize}
	\subsection{Invariant}
	An invariant is a characteristic of remaining unchanged after operations or transformations.
	\subsection{Corollary}
	A deduction or inference from other proofs. A simple consequence of something else.
	\subsection{Lemma}
	A lemma is a tool for a bigger thing, often a theorem or another lemma.
	
	\section{Puzzle problem}
	Given the puzzle on the left, can you turn it into the puzzle on the right? \\
	The only legal move is sliding the square to an adjacent square, and the empty one is a gap you can push around.\\\\
	from
	\begin{tabular}{ |c|c|c| } 
		\hline
		A & B & C \\
		\hline
		D & E & F \\ 
		\hline
		H & G & \\ 
		\hline
	\end{tabular}
	to
	\begin{tabular}{ |c|c|c| } 
		\hline
		A & B & C \\
		\hline
		D & E & F \\ 
		\hline
		G & H & \\ 
		\hline
	\end{tabular}
	
	\pagebreak
	
	\subsection{Induction}
	\textbf{Theorem}: there is no sequence of legal moves to invert G and H and also keep the other letters in their original position.\\\\
	There are two types of moves: row moves and column moves.\\
	\begin{center}
		Row move:\\
		$ $\\
		\begin{tabular}{ |c|c|c| } 
			\hline
			A & B & C \\
			\hline
			D & G & \\ 
			\hline
			E & F & H\\ 
			\hline
		\end{tabular} $=>$
		\begin{tabular}{ |c|c|c| } 
			\hline
			A & B & C \\
			\hline
			D &  & G \\ 
			\hline
			E & F & H\\ 
			\hline
		\end{tabular}
	\end{center}
	\textbf{Lemma 1}: Row moves don't change the order of the items. In the above example, even after moving, the order is still [A, B, C, D, G, E, F, H].\\
	\textbf{Proof}: In a row move, we move an item from cell $i$ to $i + 1$ or $i - 1$. Whichever direction we move to, there must be an empty space, so it can never change the order of our sequence. Nothing else moves. \qedsymbol
	
	\begin{center}
		Column move:\\
		$ $\\
		\begin{tabular}{ |c|c|c| } 
			\hline
			A & B & C \\
			\hline
			D & F & \\ 
			\hline
			H & E & G\\ 
			\hline
		\end{tabular} $=>$
		\begin{tabular}{ |c|c|c| } 
			\hline
			A & B & C \\
			\hline
			D & F  & G \\ 
			\hline
			H & E & \\ 
			\hline
		\end{tabular}
		\\
		$ $\\
		$ $\\
		\begin{tabular}{ |c|c|c| } 
			\hline
			A & B & C \\
			\hline
			D &  & G \\ 
			\hline
			H & E & F\\ 
			\hline
		\end{tabular} $=>$
		\begin{tabular}{ |c|c|c| } 
			\hline
			A &  & C \\
			\hline
			D & B  & G \\ 
			\hline
			H & E & F\\ 
			\hline
		\end{tabular}
	\end{center}
	\textbf{Lemma 2}: A column move changes the relative order of precisely 2 pairs of items.\\
	\textbf{Proof}: In a column move, we move an item in cell $i$ to a blank spot in cell $i + 3$ or $i - 3$. When an item moves 3 positions, it changes relative order with two other items ($i - 1, i - 2$ or $i + 1, i + 2$). \qedsymbol
	\\
	\\
	An \textbf{inversion} is when a letter $L_1$ precedes $L_2$ in the alphabet, but is after $L_2$ in the puzzle. In the initial case, we have one inversion of G and H.
	
	\pagebreak
	
	\textbf{Lemma 3}: During a move, the number of inversions can only increase by two, decrease by two, or stay the same.
	
	\textbf{Proof}: In a row move, there are no changes (by lemma 1). In a column move, there are three cases:
	\begin{itemize}
		\item Both pair were in order prior to the move: number of inversions goes up by two
		\item Both pairs were inverted prior to the move: number of inversions goes down by two
		\item One pair was inverted, the other wasn't: number of inversions stays the same
	\end{itemize}
	\textbf{Corollary 1}: The parity (even/odd) of the number of inversions does not change.\\
	\textbf{NOTE:} Pairs are relative to every other letter! Example: the pairs are [A, B], [A, C], ..., [B, C], [B, D], ... [C, D], [C, E], ..., [G, H].
	\\
	
	\textbf{Lemma 4}: In every state reachable from the following puzzle, the parity of number of inversions is odd:
	\\
	
	\begin{tabular}{ |c|c|c| } 
		\hline
		A & B & C \\
		\hline
		D & E & F \\ 
		\hline
		H & G & \\ 
		\hline
	\end{tabular}
	\\
	\\
	
	\textbf{Proof by induction}: $P(n)$ = after any sequence of $n$ moves from the start state, the parity of number of inversions is odd.
	\\
	
	\textbf{Base case}:  for $n = 0$, the parity of \# inversions is the same because we haven't made a move.
	\\
	
	\textbf{Inductive step}: for $n \geq 0$, show $P(n) => P(n + 1)$. By assuming $P(n)$ is true, we know the parity up until the move $n$ is odd. By \textbf{Corollary 1}, we know that after a single move, the parity of \# inversions doesn't change. Therefore $P(n) => P(n + 1)$.
	\\
	\subsection{Proof of theorem}
	The parity of the number of inversions in the desired state is even ($0$) --- it's an \textbf{invariant}. By \textbf{Lemma 4}, the desired state cannot be reached from the start state because its parity is odd.
	
	\pagebreak
	
	\section{Strong induction}
	Let $P(n)$ be any predicate. If $P(0)$ is true, and $\forall\ n \left( P\left(0\right), P\left(1\right), ..., P\left(n\right) \right) => P\left(n + 1\right) $, then $\forall\ n\ \ P(n)$ is true.
	
	\section{Unstacking game}
	You have a stack of 8 blocks. You can divide this stack into smaller stacks, and you get $x * y$ points, where $x$ is the amount of blocks in the left stack, and $y$ is the amount of blocks in the right stack. You keep going until you have 8 stacks of height 1, and your score is the sum of all your points.
	
	\begin{forest}
		for tree={circle,draw, l sep=20pt}
		[8
			[5]
			[3, edge label={node[midway, right] {Points: $5 * 3 = 15$}}
				[2]
				[1, edge label={node[midway, right] {Points: $2 * 1 = 2$}}]
			]
		]
	\end{forest}
	
	Current score: $15 + 2 = 17$
	
	\subsection{Theorem}
	All strategies for the n-block game produce the same score $S(n)$.
	
	\pagebreak
	
	\subsection{Proof by strong induction}
	\subsubsection{Predicate}
	$P(n) = S(n)$
	\subsubsection{Base case}
	for $n = 1$, $S(1) = 0$ because there are no moves.
	\subsubsection{Inductive step}
	We assume $P(1), P(2), ..., P(n)$ to be true in order to prove $P(n + 1)$.
	
	\begin{forest}
		for tree={circle,draw, l sep=20pt}
		[n + 1
			[k
				[P(k)]
			]
			[n + 1 - k, edge label={node[midway, right] {Points: $k(n + 1 - k) + P(k) + P(n + 1 - k)$}}
				[P(n + 1 - k)]
			]
		]
	\end{forest}
	\\
	\\
	We can't proceed with just this information, so let's make a stronger assumption. Let's give $S(n)$ a formula: $S(n) = \frac{n(n - 1)}{2}$.\\
	For $n = 8$, $S(8) = 28$, which is correct.\\
	For $n = 2$, $S(2) = 1$, which is correct.\\
	For $n = 3$, $S(3) = 3$, which is correct.\\
	For $n = 1$, $S(1) = 0$, which is correct.
	
	\pagebreak
	
	Now, let's go back to our initial problem and replace $P(n)$ with $\frac{n(n - 1)}{2}$:
	\begin{equation}
		k(n + 1 - k) + P(k) + P(n + 1 - k)
	\end{equation}
	\begin{equation}
		k(n + 1 - k) + \frac{k(k - 1)}{2} + \frac{(n + 1 - k)(n - k)}{2}
	\end{equation}
	\begin{equation}
		\frac{\cancel{2kn} + \cancel{2k} - \cancel{2k^2} + \cancel{k^2} - \cancel{k} + (n + 1)n - \cancel{kn} - \cancel{k} + \cancel{k^2} - \cancel{kn}}{2}
	\end{equation}
	\begin{equation}
		\frac{(n + 1)n}{2}
	\end{equation}
	\begin{equation}
		\frac{(n + 1)n}{2} = S(n + 1)
	\end{equation}
	\qed
	\\
	Just like this, we proved $P(n)$ to be true.
\end{document}