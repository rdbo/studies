\documentclass{article}
\usepackage{amsmath}
\usepackage{amsthm}
\begin{document}
	\section{Concepts}
	\begin{itemize}
		\item \textbf{Number theory} is the study of the integers. It's an old science and only recently it got a major use for cryptography.
		\item \textbf{Cryptography} is the study and practice of hiding numbers.
		\item $gcd(a, b)$ is the greatest common divisor between $a$ and $b$.
		\item $a$ and $b$ are relatively prime because $gcd(a, b) = 1$.
	\end{itemize}
	
	\section{4 gallon jug problem}
	You have a 5 gallon jug and a 3 gallon jug. The objective is to fill the 5 gallon jug to exactly 4 gallons of water. To do that, you can fill up the 3 gallon jug, pour it into the 5 gallon jug. Now you fill the 3 gallon jug again, and pour it on the 5 gallon jug until it's filled to the top. This leaves the 3 gallon jug filled 1 gallon of water, and the 5 gallon jug completely full. You pour out the water of the 5 gallon jug, and pour the 3 gallon jug in the 5 gallon jug. After that, you just have to fill the 3 gallon jug and pour it again on the 5 gallon jug and you will have exactly 4 gallons of water in it.
	
	\subsection{The logic behind the problem}
	$m | n$ if $\exists\ k\ \ k * m = n$\\
	$m | 0$ always
	
	\subsection{Theorem}
	You have an $a$ gallon jug and a $b$ gallon jug, and you want to fill the $b$ gallon jug to $q$ gallons.
	
	$a \leq b$
	
	If $m | a$ and $m | b$, then $m|$any result we can get
	\\
	For the 3 and 5 gallon case, our $m$ is $1$, because it's the only number that divides 3 and 5. For a 3 and 6 gallon case, the $m$ would be $3$, and therefore you cannot fill it up to $4$ gallons, since 3 does not divide 4 (theorem).
	
	\pagebreak
	
	\subsection{State machine}
	\textbf{States}: $(x, y)$ where $x$ is the number of gallons in the "$a$-jug", and $y$ is the number of gallons in the "$b$-jug".\\
	\textbf{Transitions}:
	\begin{itemize}
		\item \textbf{Emptying}:
		
		$(x, y) \rightarrow (0, y)$:  empty the "a-jug"
		
		$(x, y) \rightarrow (x, 0)$:  empty the "b-jug"
		
		\item \textbf{Filling}:
		
		$(x, y) \rightarrow (a, y)$: fill the "a-jug"
		
		$(x, y) \rightarrow (x, b)$: fill the "b-jug"
		
		\item \textbf{Pouring}:
		
		$(x, y) \rightarrow (0, x + y), x + y \leq b$: add the contents of the "a-jug" to the "b-jug", if it has enough space left
		
		$(x, y) \rightarrow (x - (b - y), b), x + y - b \geq b$: add just contents of the "a-jug" that fits into the "b-jug"
		
		$(x, y) \rightarrow (x + y, 0), x + y \leq a$: add the contents of the "b-jug" to the "a-jug", if it has enough space left
		
		$(x, y) \rightarrow (a, y - (a - x)) = (a, x + y - a), x + y - a \geq a$: add just contents of the "b-jug" that fits into the "a-jug"
	\end{itemize}
	
	\subsection{Proof (by induction)}
	\textbf{Invariant}: $P(n) = $"If $(x, y)$ is the state after $n$ transitions, we can conclude that $m | x$ and $m | y$"
	\\
	\subsection{Base case}
	For $x = 0$ and $y = 0$, $m | x$ and $m | y $.
	
	\subsection{Inductive step}
	Assuming $P(n)$ is true, we can prove that $P(n + 1)$ is also true.\\
	After another transition, we have the following possible states for any jug:
	
	$0, a, b, x, y, x + y, x + y - a, x + y - b, x + y - b$
	\\
	We already assumed that $m | a$, $m | b$, $m | 0$, $m | x$, $m | y$. Any linear combination of $x, y$, such as $x + y - a$, will also give a multiple of $m$. So $m$ divides any of the states.
	\\
	\textbf{Corollary 1}: $gcd(a, b)$ divides any state for the number of gallons in a jug.
	\\
	\qed
	
	\pagebreak
	
	\section{Generic theorem for the jug problem}
	\textbf{Theorem}: Any result linear combination $L = sa + tb$ of $a$ and $b$ with $0 \leq L \leq b$ can be reached. We are still assuming $a \leq b$.
	
	Example: $4 = -2 * 3 + 2 * 5$ (this means 4 is a possible solution for 3 and 5 gallon jugs)\\
	\\
	We can make S positive by adding a new multiplication to the scene: \\
	$-2 * 3 + 2 * 5$\\
	$+\ 5 * 3 - 3 * 5$\\
	---------------------\\
	$3 * 3 - 1 * 5 = 4$
	\\
	\\
	Let's call the new $s$ and $t$ values as $s'$ and $t'$, respectively.
	\\
	\\
	\textbf{Proof}: Notice that $L = sa + tb = (s + mb)a + (t - ma)b$\\
	So $\exists\ s', t'$ with $s' > 0$\\
	\\
	Assume that $0 < L < b$.\\
	To obtain $L$ gallons, repeat the following algorithm $s'$ times:
	\begin{itemize}
		\item Fill the "a-jug"
		\item Pour it into the "b-jug". When it becomes full, empty it out and continue pouring. Keep doing this until the "a-jug" is empty.
	\end{itemize}
	
	Solving the initial problem with this algorithm gives the following results:
	\begin{itemize}
		\item First iteration: $(0, 0) \rightarrow (3, 0) \rightarrow (0, 3)$
		\item Second iteration: $(0, 3) \rightarrow (3, 3) \rightarrow (1, 5) \rightarrow (1, 0) \rightarrow (0, 1)$
		\item Third iteration: $(0, 1) \rightarrow (3, 1) \rightarrow (0, 4)$
	\end{itemize}
	We filled the "a-jug" $s'$ times. Suppose that we empty out the "b-jug" $u$ times. Let $r$ be the remainder in the "b-jug" so that $0 \leq r \leq b$ and $r = s' * a - u * b$.
	\\
	We want to affirm that $r = L$, and to do that we can:
	
	$r = s'a + t'b - t'b - ub$
	
	$L = s'a + t'b$
	
	$r = L - t'b - ub = L - b(t' + u)$
	\\
	We can do a couple of affirmations about $t'$ and $u$.
	
	$t' + u \neq 0 \implies r < 0 \lor r > b$ (not possible!)
	
	$t' + u = 0 \implies u = -t'$
	\\
	Therefore $r = L$.\\
	\qed
	
	\pagebreak
	
	\section{Euclid's algorithm}
	For every $a$ and $b$, there exists a unique quotient $q$ and a remainder $r$ such that $b = qa + r$ with $0 \leq r < a$. Note: $r = rem(b, a)$\\
	\\
	\textbf{Lemma 1}: $\gcd(a, b) = \gcd(rem(b, a), a)$\\
	\\
	Example:
	
	$\gcd(105, 224) = \gcd(rem(105, 224), 105)$
	
	$224 = 2 * 105 + 14 \implies rem(105, 224) = 14$
	
	$\gcd(105, 224) = \gcd(14, 105) = \gcd(rem(14, 105), 105)$
	
	$105 = 7 * 14 + 7 \implies rem(14, 105) = 7$
	
	$gcd(105, 224) = gcd(rem(14, 7), 7)$
	
	$gcd(0, 7) = 7$
	
	$gcd(105, 224) = 7$
\end{document}